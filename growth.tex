% Options for packages loaded elsewhere
\PassOptionsToPackage{unicode}{hyperref}
\PassOptionsToPackage{hyphens}{url}
%
\documentclass[
]{article}
\usepackage{amsmath,amssymb}
\usepackage{lmodern}
\usepackage{iftex}
\ifPDFTeX
  \usepackage[T1]{fontenc}
  \usepackage[utf8]{inputenc}
  \usepackage{textcomp} % provide euro and other symbols
\else % if luatex or xetex
  \usepackage{unicode-math}
  \defaultfontfeatures{Scale=MatchLowercase}
  \defaultfontfeatures[\rmfamily]{Ligatures=TeX,Scale=1}
\fi
% Use upquote if available, for straight quotes in verbatim environments
\IfFileExists{upquote.sty}{\usepackage{upquote}}{}
\IfFileExists{microtype.sty}{% use microtype if available
  \usepackage[]{microtype}
  \UseMicrotypeSet[protrusion]{basicmath} % disable protrusion for tt fonts
}{}
\makeatletter
\@ifundefined{KOMAClassName}{% if non-KOMA class
  \IfFileExists{parskip.sty}{%
    \usepackage{parskip}
  }{% else
    \setlength{\parindent}{0pt}
    \setlength{\parskip}{6pt plus 2pt minus 1pt}}
}{% if KOMA class
  \KOMAoptions{parskip=half}}
\makeatother
\usepackage{xcolor}
\IfFileExists{xurl.sty}{\usepackage{xurl}}{} % add URL line breaks if available
\IfFileExists{bookmark.sty}{\usepackage{bookmark}}{\usepackage{hyperref}}
\hypersetup{
  pdftitle={EDS 230/ESM 232 Growth Model Assignment},
  pdfauthor={Genevieve Chiong, Halina Do-Linh, Mia Forsline},
  hidelinks,
  pdfcreator={LaTeX via pandoc}}
\urlstyle{same} % disable monospaced font for URLs
\usepackage[margin=1in]{geometry}
\usepackage{color}
\usepackage{fancyvrb}
\newcommand{\VerbBar}{|}
\newcommand{\VERB}{\Verb[commandchars=\\\{\}]}
\DefineVerbatimEnvironment{Highlighting}{Verbatim}{commandchars=\\\{\}}
% Add ',fontsize=\small' for more characters per line
\usepackage{framed}
\definecolor{shadecolor}{RGB}{248,248,248}
\newenvironment{Shaded}{\begin{snugshade}}{\end{snugshade}}
\newcommand{\AlertTok}[1]{\textcolor[rgb]{0.94,0.16,0.16}{#1}}
\newcommand{\AnnotationTok}[1]{\textcolor[rgb]{0.56,0.35,0.01}{\textbf{\textit{#1}}}}
\newcommand{\AttributeTok}[1]{\textcolor[rgb]{0.77,0.63,0.00}{#1}}
\newcommand{\BaseNTok}[1]{\textcolor[rgb]{0.00,0.00,0.81}{#1}}
\newcommand{\BuiltInTok}[1]{#1}
\newcommand{\CharTok}[1]{\textcolor[rgb]{0.31,0.60,0.02}{#1}}
\newcommand{\CommentTok}[1]{\textcolor[rgb]{0.56,0.35,0.01}{\textit{#1}}}
\newcommand{\CommentVarTok}[1]{\textcolor[rgb]{0.56,0.35,0.01}{\textbf{\textit{#1}}}}
\newcommand{\ConstantTok}[1]{\textcolor[rgb]{0.00,0.00,0.00}{#1}}
\newcommand{\ControlFlowTok}[1]{\textcolor[rgb]{0.13,0.29,0.53}{\textbf{#1}}}
\newcommand{\DataTypeTok}[1]{\textcolor[rgb]{0.13,0.29,0.53}{#1}}
\newcommand{\DecValTok}[1]{\textcolor[rgb]{0.00,0.00,0.81}{#1}}
\newcommand{\DocumentationTok}[1]{\textcolor[rgb]{0.56,0.35,0.01}{\textbf{\textit{#1}}}}
\newcommand{\ErrorTok}[1]{\textcolor[rgb]{0.64,0.00,0.00}{\textbf{#1}}}
\newcommand{\ExtensionTok}[1]{#1}
\newcommand{\FloatTok}[1]{\textcolor[rgb]{0.00,0.00,0.81}{#1}}
\newcommand{\FunctionTok}[1]{\textcolor[rgb]{0.00,0.00,0.00}{#1}}
\newcommand{\ImportTok}[1]{#1}
\newcommand{\InformationTok}[1]{\textcolor[rgb]{0.56,0.35,0.01}{\textbf{\textit{#1}}}}
\newcommand{\KeywordTok}[1]{\textcolor[rgb]{0.13,0.29,0.53}{\textbf{#1}}}
\newcommand{\NormalTok}[1]{#1}
\newcommand{\OperatorTok}[1]{\textcolor[rgb]{0.81,0.36,0.00}{\textbf{#1}}}
\newcommand{\OtherTok}[1]{\textcolor[rgb]{0.56,0.35,0.01}{#1}}
\newcommand{\PreprocessorTok}[1]{\textcolor[rgb]{0.56,0.35,0.01}{\textit{#1}}}
\newcommand{\RegionMarkerTok}[1]{#1}
\newcommand{\SpecialCharTok}[1]{\textcolor[rgb]{0.00,0.00,0.00}{#1}}
\newcommand{\SpecialStringTok}[1]{\textcolor[rgb]{0.31,0.60,0.02}{#1}}
\newcommand{\StringTok}[1]{\textcolor[rgb]{0.31,0.60,0.02}{#1}}
\newcommand{\VariableTok}[1]{\textcolor[rgb]{0.00,0.00,0.00}{#1}}
\newcommand{\VerbatimStringTok}[1]{\textcolor[rgb]{0.31,0.60,0.02}{#1}}
\newcommand{\WarningTok}[1]{\textcolor[rgb]{0.56,0.35,0.01}{\textbf{\textit{#1}}}}
\usepackage{graphicx}
\makeatletter
\def\maxwidth{\ifdim\Gin@nat@width>\linewidth\linewidth\else\Gin@nat@width\fi}
\def\maxheight{\ifdim\Gin@nat@height>\textheight\textheight\else\Gin@nat@height\fi}
\makeatother
% Scale images if necessary, so that they will not overflow the page
% margins by default, and it is still possible to overwrite the defaults
% using explicit options in \includegraphics[width, height, ...]{}
\setkeys{Gin}{width=\maxwidth,height=\maxheight,keepaspectratio}
% Set default figure placement to htbp
\makeatletter
\def\fps@figure{htbp}
\makeatother
\setlength{\emergencystretch}{3em} % prevent overfull lines
\providecommand{\tightlist}{%
  \setlength{\itemsep}{0pt}\setlength{\parskip}{0pt}}
\setcounter{secnumdepth}{-\maxdimen} % remove section numbering
\ifLuaTeX
  \usepackage{selnolig}  % disable illegal ligatures
\fi

\title{EDS 230/ESM 232 Growth Model Assignment}
\author{Genevieve Chiong, Halina Do-Linh, Mia Forsline}
\date{2022-05-19}

\begin{document}
\maketitle

\hypertarget{set-up}{%
\section{Set up}\label{set-up}}

\hypertarget{introduction}{%
\section{Introduction}\label{introduction}}

Consider the following model of forest growth (where forest size in
measured in units of carbon (C)):

\begin{itemize}
\item
  dC/dt = r ∗ C for forests

  \begin{itemize}
  \item
    where C is below a threshold canopy closure
  \item
    and \emph{r} is the early exponential growth rate
  \end{itemize}
\item
  dC/dt = g ∗ (1−C/K) for forests

  \begin{itemize}
  \item
    where C is at or above the threshold canopy closure
  \item
    and \emph{g} is the linear growth rate
  \end{itemize}
\item
  and \emph{K} is a carrying capacity in units of carbon.
\end{itemize}

The size of the forest (C), canopy closure threshold, and carrying
capacity (K) are all in units of carbon.

You could think of the canopy closure threshold as the size of the
forest at which growth rates change from exponential to linear.

You can think of \emph{r}, as early exponential growth rate and \emph{g}
as the linear growth rate once canopy closure has been reached

\hypertarget{implement-this-model-in-r-as-a-differential-equation}{%
\section{1. Implement this model in R (as a differential
equation)}\label{implement-this-model-in-r-as-a-differential-equation}}

\begin{Shaded}
\begin{Highlighting}[]
\FunctionTok{source}\NormalTok{(}\FunctionTok{here}\NormalTok{(}\StringTok{"R"}\NormalTok{, }\StringTok{"forest\_growth.R"}\NormalTok{))}
\FunctionTok{source}\NormalTok{(}\FunctionTok{here}\NormalTok{(}\StringTok{"R"}\NormalTok{, }\StringTok{"compute\_metrics.R"}\NormalTok{))}

\NormalTok{forest\_growth}
\end{Highlighting}
\end{Shaded}

\begin{verbatim}
## function (time, C, parms) 
## {
##     if (C < parms$threshold) {
##         dC_dt = parms$r * C
##     }
##     else {
##         dC_dt = parms$g * (1 - (C/parms$K))
##     }
##     return(list(dC_dt))
## }
\end{verbatim}

\begin{Shaded}
\begin{Highlighting}[]
\NormalTok{compute\_metrics}
\end{Highlighting}
\end{Shaded}

\begin{verbatim}
## function (result) 
## {
##     maxpop = max(result$P)
##     idx = which.max(result$P)
##     meanpop = mean(result$P)
##     return(list(maxpop = maxpop, meanpop = meanpop))
## }
\end{verbatim}

\hypertarget{run-the-model-for-300-years-using-the-ode-solver-starting-with-an-initial-forest-size-of-10-kgc-and-using-the-following-parameters}{%
\section{2. Run the model for 300 years (using the ODE solver) starting
with an initial forest size of 10 kg/C, and using the following
parameters:}\label{run-the-model-for-300-years-using-the-ode-solver-starting-with-an-initial-forest-size-of-10-kgc-and-using-the-following-parameters}}

\begin{itemize}
\item
  canopy closure threshold of 50 kgC
\item
  \emph{K} = 250 kg C (carrying capacity)
\item
  \emph{r} = 0.01 (exponential growth rate before before canopy closure)
\item
  \emph{g} = 2 kg/year (linear growth rate after canopy closure)
\end{itemize}

\begin{Shaded}
\begin{Highlighting}[]
\CommentTok{\#set up parameters }
\NormalTok{C }\OtherTok{=} \DecValTok{10} \CommentTok{\#forest size}
\NormalTok{time }\OtherTok{=} \FunctionTok{seq}\NormalTok{(}\AttributeTok{from =} \DecValTok{1}\NormalTok{, }\AttributeTok{to =} \DecValTok{300}\NormalTok{) }\CommentTok{\#run the model for 300 years}
\NormalTok{threshold }\OtherTok{=} \DecValTok{50} \CommentTok{\#canopy closure }
\NormalTok{K }\OtherTok{=} \DecValTok{250}
\NormalTok{r }\OtherTok{=} \FloatTok{0.01} \CommentTok{\#exponential growth rate}
\NormalTok{g }\OtherTok{=} \DecValTok{2} \CommentTok{\#linear growth rate }

\CommentTok{\#create parameters list}
\NormalTok{parms }\OtherTok{=} \FunctionTok{list}\NormalTok{(}\AttributeTok{time =}\NormalTok{ time,}
             \AttributeTok{threshold =}\NormalTok{ threshold,}
             \AttributeTok{K =}\NormalTok{ K,}
             \AttributeTok{r =}\NormalTok{ r,}
             \AttributeTok{g =}\NormalTok{ g)}

\CommentTok{\#apply solver {-} we input the differential equation and the ODE solver integrates it }
\NormalTok{results }\OtherTok{=} \FunctionTok{ode}\NormalTok{(C, time, forest\_growth, parms)}

\FunctionTok{head}\NormalTok{(results)}
\end{Highlighting}
\end{Shaded}

\begin{verbatim}
##      time        1
## [1,]    1 10.00000
## [2,]    2 10.10050
## [3,]    3 10.20202
## [4,]    4 10.30455
## [5,]    5 10.40811
## [6,]    6 10.51271
\end{verbatim}

\begin{Shaded}
\begin{Highlighting}[]
\CommentTok{\# add more meaningful names}
\FunctionTok{colnames}\NormalTok{(results) }\OtherTok{=} \FunctionTok{c}\NormalTok{(}\StringTok{"year"}\NormalTok{,}\StringTok{"P"}\NormalTok{)}
\end{Highlighting}
\end{Shaded}

\hypertarget{graph-the-results.}{%
\subsection{Graph the results.}\label{graph-the-results.}}

\begin{Shaded}
\begin{Highlighting}[]
\FunctionTok{ggplot}\NormalTok{(}\FunctionTok{as.data.frame}\NormalTok{(results), }\FunctionTok{aes}\NormalTok{(year, P)) }\SpecialCharTok{+} 
  \FunctionTok{geom\_point}\NormalTok{() }\SpecialCharTok{+} 
  \FunctionTok{labs}\NormalTok{(}\AttributeTok{y=}\StringTok{"Forest Population"}\NormalTok{, }
       \AttributeTok{x =} \StringTok{"Years"}\NormalTok{) }\SpecialCharTok{+}
  \FunctionTok{theme\_classic}\NormalTok{() }\SpecialCharTok{+}
\FunctionTok{geom\_point}\NormalTok{(}\AttributeTok{col =} \StringTok{"darkgreen"}\NormalTok{,}
             \AttributeTok{size =} \DecValTok{1}\NormalTok{) }\SpecialCharTok{+} 
  \FunctionTok{ggtitle}\NormalTok{(}\AttributeTok{label =} \StringTok{"Forest Population after 300 Years of Growth"}\NormalTok{,}
          \AttributeTok{subtitle =} \StringTok{""}\NormalTok{) }\SpecialCharTok{+}
  \FunctionTok{labs}\NormalTok{(}\AttributeTok{y =}\StringTok{"Population"}\NormalTok{, }
       \AttributeTok{x =} \StringTok{"Years"}\NormalTok{) }\SpecialCharTok{+} 
  \FunctionTok{geom\_vline}\NormalTok{(}\AttributeTok{xintercept =} \DecValTok{50}\NormalTok{) }\SpecialCharTok{+}
  \FunctionTok{theme\_classic}\NormalTok{() }\SpecialCharTok{+}
  \FunctionTok{scale\_x\_continuous}\NormalTok{(}\AttributeTok{breaks =} \FunctionTok{seq}\NormalTok{(}\DecValTok{0}\NormalTok{, }\DecValTok{300}\NormalTok{, }\AttributeTok{by =} \DecValTok{25}\NormalTok{)) }\SpecialCharTok{+}
  \FunctionTok{scale\_y\_continuous}\NormalTok{(}\AttributeTok{breaks =} \FunctionTok{seq}\NormalTok{(}\DecValTok{0}\NormalTok{, }\DecValTok{200}\NormalTok{, }\AttributeTok{by =} \DecValTok{25}\NormalTok{))}
\end{Highlighting}
\end{Shaded}

\includegraphics{growth_files/figure-latex/unnamed-chunk-3-1.pdf}

\hypertarget{run-a-sobol-sensitivity-analysis-that-explores-how-the-estimated-maximum-and-mean-forest-size-e.g-maximum-and-mean-values-of-c-over-the-300-years-varies-with-the-pre-canopy-closure-growth-rate-r-and-post-canopy-closure-growth-rate-g-and-canopy-closure-threshold-and-carrying-capacityk}{%
\section{\texorpdfstring{3. Run a sobol sensitivity analysis that
explores how the estimated maximum and mean forest size (e.g maximum and
mean values of C over the 300 years) varies with the pre canopy closure
growth rate (\emph{r}) and post-canopy closure growth rate (\emph{g})
and canopy closure threshold and carrying
capacity(\emph{K})}{3. Run a sobol sensitivity analysis that explores how the estimated maximum and mean forest size (e.g maximum and mean values of C over the 300 years) varies with the pre canopy closure growth rate (r) and post-canopy closure growth rate (g) and canopy closure threshold and carrying capacity(K)}}\label{run-a-sobol-sensitivity-analysis-that-explores-how-the-estimated-maximum-and-mean-forest-size-e.g-maximum-and-mean-values-of-c-over-the-300-years-varies-with-the-pre-canopy-closure-growth-rate-r-and-post-canopy-closure-growth-rate-g-and-canopy-closure-threshold-and-carrying-capacityk}}

Assume that parameters are all normally distributed with means as given
above and standard deviation of 10\% of mean value.

\begin{Shaded}
\begin{Highlighting}[]
\CommentTok{\#set parameters }
\NormalTok{C }\OtherTok{=} \DecValTok{10} \CommentTok{\#forest size}

\CommentTok{\#number of samples }
\NormalTok{np }\OtherTok{=} \DecValTok{100} 

\CommentTok{\#create first sample parameters from normal distributions }
\NormalTok{r }\OtherTok{=} \FunctionTok{rnorm}\NormalTok{(}\AttributeTok{mean =} \FloatTok{0.01}\NormalTok{, }\AttributeTok{sd =}\NormalTok{ r}\SpecialCharTok{*}\NormalTok{.}\DecValTok{10}\NormalTok{, }\AttributeTok{n =}\NormalTok{ np)}
\NormalTok{g }\OtherTok{=} \FunctionTok{rnorm}\NormalTok{(}\AttributeTok{mean =} \DecValTok{2}\NormalTok{, }\AttributeTok{sd =}\NormalTok{ g}\SpecialCharTok{*}\NormalTok{.}\DecValTok{10}\NormalTok{, }\AttributeTok{n =}\NormalTok{ np)}
\NormalTok{threshold }\OtherTok{=} \FunctionTok{rnorm}\NormalTok{(}\AttributeTok{mean =} \DecValTok{50}\NormalTok{, }\AttributeTok{sd =}\NormalTok{ threshold}\SpecialCharTok{*}\NormalTok{.}\DecValTok{10}\NormalTok{, }\AttributeTok{n =}\NormalTok{ np)}
\NormalTok{K }\OtherTok{=} \FunctionTok{rnorm}\NormalTok{(}\AttributeTok{mean =} \DecValTok{250}\NormalTok{, }\AttributeTok{sd =}\NormalTok{ K}\SpecialCharTok{*}\NormalTok{.}\DecValTok{10}\NormalTok{, }\AttributeTok{n =}\NormalTok{ np)}

\CommentTok{\#create the first dataframe }
\NormalTok{X1 }\OtherTok{=} \FunctionTok{cbind.data.frame}\NormalTok{(}\AttributeTok{r =}\NormalTok{ r, }\AttributeTok{g =}\NormalTok{ g, }\AttributeTok{threshold =}\NormalTok{ threshold, }\AttributeTok{K =}\NormalTok{ K)}

\CommentTok{\#create second sample parameters from normal distributions (this is just how sobol works)}
\NormalTok{r }\OtherTok{=} \FunctionTok{rnorm}\NormalTok{(}\AttributeTok{mean =} \FloatTok{0.01}\NormalTok{, }\AttributeTok{sd =}\NormalTok{ r}\SpecialCharTok{*}\NormalTok{.}\DecValTok{10}\NormalTok{, }\AttributeTok{n =}\NormalTok{ np)}
\NormalTok{g }\OtherTok{=} \FunctionTok{rnorm}\NormalTok{(}\AttributeTok{mean =} \DecValTok{2}\NormalTok{, }\AttributeTok{sd =}\NormalTok{ g}\SpecialCharTok{*}\NormalTok{.}\DecValTok{10}\NormalTok{, }\AttributeTok{n =}\NormalTok{ np)}
\NormalTok{threshold }\OtherTok{=} \FunctionTok{rnorm}\NormalTok{(}\AttributeTok{mean =} \DecValTok{50}\NormalTok{, }\AttributeTok{sd =}\NormalTok{ threshold}\SpecialCharTok{*}\NormalTok{.}\DecValTok{10}\NormalTok{, }\AttributeTok{n =}\NormalTok{ np)}
\NormalTok{K }\OtherTok{=} \FunctionTok{rnorm}\NormalTok{(}\AttributeTok{mean =} \DecValTok{250}\NormalTok{, }\AttributeTok{sd =}\NormalTok{ K}\SpecialCharTok{*}\NormalTok{.}\DecValTok{10}\NormalTok{, }\AttributeTok{n =}\NormalTok{ np)}

\CommentTok{\#create the second dataframe }
\NormalTok{X2 }\OtherTok{=} \FunctionTok{cbind.data.frame}\NormalTok{(}\AttributeTok{r =}\NormalTok{ r, }\AttributeTok{g =}\NormalTok{ g, }\AttributeTok{threshold =}\NormalTok{ threshold, }\AttributeTok{K =}\NormalTok{ K)}
\end{Highlighting}
\end{Shaded}

\begin{Shaded}
\begin{Highlighting}[]
\CommentTok{\#create our sobel object and get sets of parameters for running the model}
\NormalTok{sens\_P }\OtherTok{=} \FunctionTok{sobolSalt}\NormalTok{(}\AttributeTok{model =} \ConstantTok{NULL}\NormalTok{, X1, X2, }\AttributeTok{nboot =} \DecValTok{300}\NormalTok{)}

\FunctionTok{colnames}\NormalTok{(sens\_P}\SpecialCharTok{$}\NormalTok{X) }\OtherTok{=} \FunctionTok{c}\NormalTok{(}\StringTok{"r"}\NormalTok{,}
                       \StringTok{"g"}\NormalTok{,}
                       \StringTok{"threshold"}\NormalTok{,}
                       \StringTok{"K"}\NormalTok{)}

\CommentTok{\#our parameter sets are}
\FunctionTok{head}\NormalTok{(sens\_P}\SpecialCharTok{$}\NormalTok{X)}
\end{Highlighting}
\end{Shaded}

\begin{verbatim}
##                r        g threshold        K
## [1,] 0.009214390 2.231618  50.25835 296.8523
## [2,] 0.010027606 2.308116  46.94140 264.0170
## [3,] 0.009350603 1.934701  61.46352 244.3394
## [4,] 0.011132196 1.954695  48.89945 195.1303
## [5,] 0.009980618 2.016259  49.79585 292.4689
## [6,] 0.010110885 2.131441  54.66925 275.6741
\end{verbatim}

\begin{Shaded}
\begin{Highlighting}[]
\CommentTok{\# define a wrapper function to do everything we need {-} run solver and compute metrics {-} and send back results for each parameter}

\NormalTok{p\_wrapper }\OtherTok{=} \ControlFlowTok{function}\NormalTok{(r, g, threshold, K, Pinitial, simtimes, func) \{}
\NormalTok{    parms }\OtherTok{=} \FunctionTok{list}\NormalTok{(}\AttributeTok{r =}\NormalTok{ r,}
                 \AttributeTok{g =}\NormalTok{ g,}
                 \AttributeTok{threshold =}\NormalTok{ threshold,}
                 \AttributeTok{K =}\NormalTok{ K)}
\NormalTok{    result }\OtherTok{=} \FunctionTok{ode}\NormalTok{(}\AttributeTok{y =}\NormalTok{ Pinitial, }
                 \AttributeTok{times =}\NormalTok{ simtimes, }
                 \AttributeTok{func =}\NormalTok{ func, }
                 \AttributeTok{parms =}\NormalTok{ parms) }
    \FunctionTok{colnames}\NormalTok{(result) }\OtherTok{=} \FunctionTok{c}\NormalTok{(}\StringTok{"time"}\NormalTok{,}\StringTok{"P"}\NormalTok{)}
  \CommentTok{\# get metrics}
\NormalTok{  metrics }\OtherTok{=} \FunctionTok{compute\_metrics}\NormalTok{(}\FunctionTok{as.data.frame}\NormalTok{(result))}
  \FunctionTok{return}\NormalTok{(metrics)}
\NormalTok{\}}
\end{Highlighting}
\end{Shaded}

\begin{Shaded}
\begin{Highlighting}[]
\CommentTok{\# reiterate parameters }
\NormalTok{Pinitial }\OtherTok{=} \DecValTok{10} \CommentTok{\#forest size (initial population in units of kg C)}
\NormalTok{simtimes }\OtherTok{=} \FunctionTok{seq}\NormalTok{(}\AttributeTok{from =} \DecValTok{1}\NormalTok{, }\AttributeTok{to =} \DecValTok{300}\NormalTok{) }\CommentTok{\#run the model for 300 years}
\NormalTok{func }\OtherTok{=}\NormalTok{ forest\_growth}

\CommentTok{\# now use pmap as we did before}
\NormalTok{allresults }\OtherTok{=} \FunctionTok{as.data.frame}\NormalTok{(sens\_P}\SpecialCharTok{$}\NormalTok{X) }\SpecialCharTok{\%\textgreater{}\%} 
  \FunctionTok{pmap}\NormalTok{(p\_wrapper, }
       \AttributeTok{Pinitial =}\NormalTok{ Pinitial, }
       \AttributeTok{simtimes =}\NormalTok{ simtimes, }
       \AttributeTok{func =}\NormalTok{ func)}
\end{Highlighting}
\end{Shaded}

\begin{Shaded}
\begin{Highlighting}[]
\CommentTok{\# extract out results from pmap into a data frame}
\NormalTok{allres }\OtherTok{=}\NormalTok{ allresults }\SpecialCharTok{\%\textgreater{}\%} \FunctionTok{map\_dfr}\NormalTok{(}\StringTok{\textasciigrave{}}\AttributeTok{[}\StringTok{\textasciigrave{}}\NormalTok{,}\FunctionTok{c}\NormalTok{(}\StringTok{"maxpop"}\NormalTok{,}\StringTok{"meanpop"}\NormalTok{))}
\end{Highlighting}
\end{Shaded}

\hypertarget{graph-the-results}{%
\subsection{Graph the results}\label{graph-the-results}}

\begin{Shaded}
\begin{Highlighting}[]
\CommentTok{\# create boxplots}
\NormalTok{tmp }\OtherTok{=}\NormalTok{ allres }\SpecialCharTok{\%\textgreater{}\%} \FunctionTok{gather}\NormalTok{(}\AttributeTok{key =} \StringTok{"metric"}\NormalTok{, }\AttributeTok{value =} \StringTok{"value"}\NormalTok{)}
\FunctionTok{ggplot}\NormalTok{(tmp, }\FunctionTok{aes}\NormalTok{(metric, value, }\AttributeTok{col =}\NormalTok{ metric)) }\SpecialCharTok{+} 
  \FunctionTok{geom\_boxplot}\NormalTok{() }\SpecialCharTok{+} 
  \FunctionTok{theme\_classic}\NormalTok{() }
\end{Highlighting}
\end{Shaded}

\includegraphics{growth_files/figure-latex/unnamed-chunk-9-1.pdf} \#\#\#
Graphs of sobol indices (S and T)

\begin{Shaded}
\begin{Highlighting}[]
\CommentTok{\# graph of sobol results as box plot of max forest size and a plot of the two sobol indices (S and T)}
\FunctionTok{ggplot}\NormalTok{(}\AttributeTok{data =}\NormalTok{ allres,}
       \FunctionTok{aes}\NormalTok{(}\AttributeTok{y =}\NormalTok{ maxpop)) }\SpecialCharTok{+} 
  \FunctionTok{geom\_boxplot}\NormalTok{() }\SpecialCharTok{+} 
  \FunctionTok{theme\_classic}\NormalTok{() }
\end{Highlighting}
\end{Shaded}

\includegraphics{growth_files/figure-latex/unnamed-chunk-10-1.pdf}

\begin{Shaded}
\begin{Highlighting}[]
\CommentTok{\# compute the sobol indices for maxpop}
\NormalTok{sens\_P\_maxpop }\OtherTok{=}\NormalTok{ sensitivity}\SpecialCharTok{::}\FunctionTok{tell}\NormalTok{(sens\_P, allres}\SpecialCharTok{$}\NormalTok{maxpop)}

\CommentTok{\# compute the sobol indices for meanpop}
\NormalTok{sens\_P\_meanpop }\OtherTok{=}\NormalTok{ sensitivity}\SpecialCharTok{::}\FunctionTok{tell}\NormalTok{(sens\_P, allres}\SpecialCharTok{$}\NormalTok{meanpop)}
\end{Highlighting}
\end{Shaded}

\begin{Shaded}
\begin{Highlighting}[]
\CommentTok{\# creating df of max T}
\NormalTok{max\_T }\OtherTok{\textless{}{-}} \FunctionTok{as.data.frame}\NormalTok{(sens\_P\_maxpop}\SpecialCharTok{$}\NormalTok{T) }

\NormalTok{max\_T }\OtherTok{\textless{}{-}}\NormalTok{ max\_T }\SpecialCharTok{\%\textgreater{}\%} 
  \FunctionTok{rowid\_to\_column}\NormalTok{(}\AttributeTok{var =} \StringTok{"parameter"}\NormalTok{)}

\NormalTok{max\_T[}\DecValTok{1}\NormalTok{,}\DecValTok{1}\NormalTok{] }\OtherTok{\textless{}{-}} \StringTok{"r"}
\NormalTok{max\_T[}\DecValTok{2}\NormalTok{,}\DecValTok{1}\NormalTok{] }\OtherTok{\textless{}{-}} \StringTok{"g"}
\NormalTok{max\_T[}\DecValTok{3}\NormalTok{,}\DecValTok{1}\NormalTok{] }\OtherTok{\textless{}{-}} \StringTok{"threshold"}
\NormalTok{max\_T[}\DecValTok{4}\NormalTok{,}\DecValTok{1}\NormalTok{] }\OtherTok{\textless{}{-}} \StringTok{"K"}

\CommentTok{\# creating df of max S}
\NormalTok{max\_S }\OtherTok{\textless{}{-}} \FunctionTok{as.data.frame}\NormalTok{(sens\_P\_maxpop}\SpecialCharTok{$}\NormalTok{S) }

\NormalTok{max\_S }\OtherTok{\textless{}{-}}\NormalTok{ max\_S }\SpecialCharTok{\%\textgreater{}\%} 
  \FunctionTok{rowid\_to\_column}\NormalTok{(}\AttributeTok{var =} \StringTok{"parameter"}\NormalTok{)}

\NormalTok{max\_S[}\DecValTok{1}\NormalTok{,}\DecValTok{1}\NormalTok{] }\OtherTok{\textless{}{-}} \StringTok{"r"}
\NormalTok{max\_S[}\DecValTok{2}\NormalTok{,}\DecValTok{1}\NormalTok{] }\OtherTok{\textless{}{-}} \StringTok{"g"}
\NormalTok{max\_S[}\DecValTok{3}\NormalTok{,}\DecValTok{1}\NormalTok{] }\OtherTok{\textless{}{-}} \StringTok{"threshold"}
\NormalTok{max\_S[}\DecValTok{4}\NormalTok{,}\DecValTok{1}\NormalTok{] }\OtherTok{\textless{}{-}} \StringTok{"K"}

\CommentTok{\# creating df of mean T}
\NormalTok{mean\_T }\OtherTok{\textless{}{-}} \FunctionTok{as.data.frame}\NormalTok{(sens\_P\_meanpop}\SpecialCharTok{$}\NormalTok{T) }

\NormalTok{mean\_T }\OtherTok{\textless{}{-}}\NormalTok{ mean\_T }\SpecialCharTok{\%\textgreater{}\%} 
  \FunctionTok{rowid\_to\_column}\NormalTok{(}\AttributeTok{var =} \StringTok{"parameter"}\NormalTok{)}

\NormalTok{mean\_T[}\DecValTok{1}\NormalTok{,}\DecValTok{1}\NormalTok{] }\OtherTok{\textless{}{-}} \StringTok{"r"}
\NormalTok{mean\_T[}\DecValTok{2}\NormalTok{,}\DecValTok{1}\NormalTok{] }\OtherTok{\textless{}{-}} \StringTok{"g"}
\NormalTok{mean\_T[}\DecValTok{3}\NormalTok{,}\DecValTok{1}\NormalTok{] }\OtherTok{\textless{}{-}} \StringTok{"threshold"}
\NormalTok{mean\_T[}\DecValTok{4}\NormalTok{,}\DecValTok{1}\NormalTok{] }\OtherTok{\textless{}{-}} \StringTok{"K"}

\CommentTok{\# creating df of mean S}
\NormalTok{mean\_S }\OtherTok{\textless{}{-}} \FunctionTok{as.data.frame}\NormalTok{(sens\_P\_meanpop}\SpecialCharTok{$}\NormalTok{S) }

\NormalTok{mean\_S }\OtherTok{\textless{}{-}}\NormalTok{ mean\_S }\SpecialCharTok{\%\textgreater{}\%} 
  \FunctionTok{rowid\_to\_column}\NormalTok{(}\AttributeTok{var =} \StringTok{"parameter"}\NormalTok{)}

\NormalTok{mean\_S[}\DecValTok{1}\NormalTok{,}\DecValTok{1}\NormalTok{] }\OtherTok{\textless{}{-}} \StringTok{"r"}
\NormalTok{mean\_S[}\DecValTok{2}\NormalTok{,}\DecValTok{1}\NormalTok{] }\OtherTok{\textless{}{-}} \StringTok{"g"}
\NormalTok{mean\_S[}\DecValTok{3}\NormalTok{,}\DecValTok{1}\NormalTok{] }\OtherTok{\textless{}{-}} \StringTok{"threshold"}
\NormalTok{mean\_S[}\DecValTok{4}\NormalTok{,}\DecValTok{1}\NormalTok{] }\OtherTok{\textless{}{-}} \StringTok{"K"}
\end{Highlighting}
\end{Shaded}

\begin{Shaded}
\begin{Highlighting}[]
\CommentTok{\# create S and T plots}
\NormalTok{max\_T\_plot }\OtherTok{\textless{}{-}} \FunctionTok{ggplot}\NormalTok{(max\_T, }\FunctionTok{aes}\NormalTok{(}\AttributeTok{x =}\NormalTok{ original, }\AttributeTok{y =}\NormalTok{ parameter)) }\SpecialCharTok{+}
  \FunctionTok{geom\_col}\NormalTok{(}\AttributeTok{fill =} \StringTok{"darkgreen"}\NormalTok{) }\SpecialCharTok{+}
  \FunctionTok{theme\_classic}\NormalTok{() }\SpecialCharTok{+}
  \FunctionTok{labs}\NormalTok{(}\AttributeTok{title =} \StringTok{"Max Total Sensitivity Index"}\NormalTok{,}
       \AttributeTok{x =} \StringTok{"Sobol Score"}\NormalTok{,}
       \AttributeTok{y =} \StringTok{"Parameter"}\NormalTok{)}

\NormalTok{max\_S\_plot }\OtherTok{\textless{}{-}} \FunctionTok{ggplot}\NormalTok{(max\_S, }\FunctionTok{aes}\NormalTok{(}\AttributeTok{x =}\NormalTok{ original, }\AttributeTok{y =}\NormalTok{ parameter)) }\SpecialCharTok{+}
  \FunctionTok{geom\_col}\NormalTok{(}\AttributeTok{fill =} \StringTok{"cyan4"}\NormalTok{) }\SpecialCharTok{+}
  \FunctionTok{theme\_classic}\NormalTok{() }\SpecialCharTok{+}
  \FunctionTok{labs}\NormalTok{(}\AttributeTok{title =} \StringTok{"Max First Order Sensitivity Index"}\NormalTok{,}
       \AttributeTok{x =} \StringTok{"Sobol Score"}\NormalTok{,}
       \AttributeTok{y =} \StringTok{"Parameter"}\NormalTok{)}

\NormalTok{mean\_T\_plot }\OtherTok{\textless{}{-}} \FunctionTok{ggplot}\NormalTok{(mean\_T, }\FunctionTok{aes}\NormalTok{(}\AttributeTok{x =}\NormalTok{ original, }\AttributeTok{y =}\NormalTok{ parameter)) }\SpecialCharTok{+}
  \FunctionTok{geom\_col}\NormalTok{(}\AttributeTok{fill =} \StringTok{"darkgreen"}\NormalTok{) }\SpecialCharTok{+}
  \FunctionTok{theme\_classic}\NormalTok{() }\SpecialCharTok{+}
  \FunctionTok{labs}\NormalTok{(}\AttributeTok{title =} \StringTok{"Mean Total Sensitivity Index"}\NormalTok{,}
       \AttributeTok{x =} \StringTok{"Sobol Score"}\NormalTok{,}
       \AttributeTok{y =} \StringTok{"Parameter"}\NormalTok{)}

\NormalTok{mean\_S\_plot }\OtherTok{\textless{}{-}} \FunctionTok{ggplot}\NormalTok{(mean\_S, }\FunctionTok{aes}\NormalTok{(}\AttributeTok{x =}\NormalTok{ original, }\AttributeTok{y =}\NormalTok{ parameter)) }\SpecialCharTok{+}
  \FunctionTok{geom\_col}\NormalTok{(}\AttributeTok{fill =} \StringTok{"cyan4"}\NormalTok{) }\SpecialCharTok{+}
  \FunctionTok{theme\_classic}\NormalTok{() }\SpecialCharTok{+}
  \FunctionTok{labs}\NormalTok{(}\AttributeTok{title =} \StringTok{"Mean First Order Sensitivity Index"}\NormalTok{,}
       \AttributeTok{x =} \StringTok{"Sobol Score"}\NormalTok{,}
       \AttributeTok{y =} \StringTok{"Parameter"}\NormalTok{)}

\CommentTok{\# patchwork plots}
\NormalTok{max\_T\_plot }\SpecialCharTok{+}\NormalTok{ max\_S\_plot}
\end{Highlighting}
\end{Shaded}

\includegraphics{growth_files/figure-latex/unnamed-chunk-12-1.pdf}

\begin{Shaded}
\begin{Highlighting}[]
\NormalTok{mean\_T\_plot }\SpecialCharTok{+}\NormalTok{ mean\_S\_plot}
\end{Highlighting}
\end{Shaded}

\includegraphics{growth_files/figure-latex/unnamed-chunk-12-2.pdf}

\hypertarget{in-2-3-sentences-discuss-what-the-results-of-your-simulation-might-mean-for-climate-change-impacts-on-forest-growth-e.g-think-about-what-parameters-climate-change-might-influence-.}{%
\section{In 2-3 sentences, discuss what the results of your simulation
might mean for climate change impacts on forest growth (e.g think about
what parameters climate change might influence
).}\label{in-2-3-sentences-discuss-what-the-results-of-your-simulation-might-mean-for-climate-change-impacts-on-forest-growth-e.g-think-about-what-parameters-climate-change-might-influence-.}}

closer it is 1, the more influential it is

\end{document}
